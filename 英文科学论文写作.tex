\documentclass[lettersize,journal,12pt]{IEEEtran}
\usepackage{fontspec}
\usepackage{amsmath,amsfonts}
\usepackage{algorithmic}
\usepackage{algorithm}
\usepackage{array}
\usepackage[caption=false,font=normalsize,labelfont=sf,textfont=sf]{subfig}
\usepackage{textcomp}
\usepackage{stfloats}
\usepackage{url}
\usepackage{verbatim}
\usepackage{xeCJK}
\usepackage{lettrine}
\usepackage{graphicx}
\usepackage{titling}
\usepackage{titlesec}
\usepackage{balance}
\usepackage{textcase}
\usepackage{setspace}
\setmainfont{Times New Roman}[SmallCapsFont=TeX Gyre Termes:+smcp]
% rule to break words
\hyphenation{}
\def\BibTeX{{\rm B\kern-.05em{\sc i\kern-.025em b}\kern-.08em
T\kern-.1667em\lower.7ex\hbox{E}\kern-.125emX}}
\pretitle{\begin{center}\fontsize{16}{18}\selectfont\bfseries}
\posttitle{\end{center}}
\preauthor{\begin{center}\fontsize{10}{12}\selectfont}
\postauthor{\end{center}}
\predate{\begin{center}\fontsize{10}{12}\selectfont}
\postdate{\end{center}}
\titleformat{\section}
{\filcenter\fontsize{14}{16}\bfseries\uppercase}
{\thesection}
{1em}
{}
\renewenvironment{abstract}
{\fontsize{12}{14}\textit{\textbf{\abstractname---}}\bfseries\ignorespaces}
{}
\renewenvironment{IEEEkeywords}
{\fontsize{12}{14}\textit{\textbf{Keywords---}}\bfseries\ignorespaces}{}
\begin{document}
\onehalfspacing
\title{Unveiling the PageRank Algorithm: Principles, Performance, and Enhancements}
\author{Wu Zelin, Wu Zekai, Li Pengda}

\maketitle\thispagestyle{headings}
\markboth{10225101428 吴泽霖\quad10225101429 武泽恺\quad10225101460 李鹏达}{}%

\begin{abstract}
	This is the abstract area. We should write a very nb abstract here.
\end{abstract}

\begin{IEEEkeywords}
	Keyword1, Keyword2, Keyword3
\end{IEEEkeywords}


\section{Introduction}

\subsection{Research Background}

\lettrine{W}{ith} 
the proliferation of the Internet technology, the explosively increasing number of web pages on the World Wide Web has created the demand for the web searching engines with high-efficiency and high-effectiveness. 
For the biggest search engine company Google, which held a global market share of 91.54\% until November 2023\footnote[1]{Search Engine Market Share Worldwide, Statcounter GlobalStats, 2023, https:
//gs.statcounter.com/search-engine-market-share}, it is of primary significance to develop a powerful search algorithm to provide users with the most relevant and useful results in the shortest time. 

\begin{figure}[h]
    \centering
    \includegraphics[width=2.5in]{images/fig2.png}
    \caption{Google's market share in the global search engine market from December 2022 to November 2023.}
    \label{fig1}
\end{figure}

Previous studies on web searching algorithms have proven that keyword matching algorithm\cite{ref1}, vector space model\cite{ref2}, and the Hyperlink-Induced Topic Search algorithm(HITS)\cite{ref3} were useful, efficient and effective in the early days of the Internet.

However, these algorithms have been unable to meet the growing demand of users. The users' search queries are becoming diversified and the web pages are becoming complex. 
According to the statistics, the number of web pages, a 130-fold increase over 20 years, has surged to 130 trillion in 2016, which means it takes much longer time to search for the most relevant web pages.
Besides, they only take into account the page content but ignore the graphical structure of the web pages.
To overcome the aformentioned chanllenge, Google has developed a state-of-the-art algorithm called PageRank, which is currently a fundamental algorithm in web search, redefining how we navigate web search.

\subsection{Origin of the PageRank Algorithm} 

PageRank is an algorithm developed by Larry Page and Sergey Brin in the late 1990s to measure the importance of web pages. 
Originally created for the Google search engine, it is named after one of the founder of Google Larry Page. Except for the searching use, the Google search engine also takes advantage of this algorithm to analyze the relevance and importance of web pages, considering it as one of the factors to evaluate the effectiveness of web page optimization. 

\subsection{Outline of this Paper}

In this paper, we present the following insights and research:
\begin{itemize}
	\item The foundational principles and implementation of the PageRank algorithm.(Section 3, Subsection A)
	\item Analysis of its performance and the impact of the factors that can influence the search results and page rankings.(Section 4, Subsection A)
	\item Discussion on potential enhancements and exploration of the possibility of other factors that can improve PageRank performance.(Section 4, Subsection B)
\end{itemize}

In the next section, we introduce the preliminaries and the related work of the PageRank algorithm.

\section{Related Work}

This section shows the brief concept and preliminaries of the PageRank algorithm, which is indispensable for the understanding of the following sections.

%这里需要改,改成之前的算法介绍,不要用这个
\subsection{Hyperlink-Induced Topic Search algorithm}

The Hyperlink-Induced Topic Search (HITS) algorithm is a link analysis algorithm that ranks web pages, developed by Jon Kleinberg. This algorithm assigns two scores to each page,including hub score and authority score. Hub is a web page that link to many other web pages, while authority is a web page that is linked by many hubs. This algorithm computes the hub and authority scores for every page on the World Wide Web. The hub and authority scores are computed iteratively until the ultimate structure of web pages is formed. This algorithm has been applied to many search engines, computational biology and many other fields. 

However, its limitations are exposed in many circumstances. For instance, simply classifying the web pages into two categories, it is unable to rank the web pages in a more precise way, which worsens as the number of web pages increases. To address this problem, the PageRank algorithm introduces link weights to define the importance of a hyperlink. For instance, if a web page is linked by an important web page, then its importance will also be higher.

For the large searching engine, simplicity and efficiency are significant considerations. The HITS algorithm requires multiple iterative and complex calculations, leading to its unsuitability for the current need.

\subsection{Synopsis of PageRank}

PageRank is a link analysis algorithm\cite{ref4}. It assigns a numerical weighting to each web page of all hyperlinked web pages. We call the weight value of any element E as ``PageRank of E", which is represented as symbol $\boldsymbol{PR(E)}$. 
The weight value of a web page could be affected by other factors, such as "Author Rank," which takes into account the author's reputation and authority.

\begin{figure}[h]
    \centering
    \includegraphics[width=2.5in]{images/fig3.jpeg}
    \caption{A simple illustration of PageRank, where each circle represents a web page and each arrow represents a hyperlink. The size of each circle represents the importance of each web page.}
    \label{fig3}
\end{figure}

A web page's PageRank is created by all World Wide Web pages as nodes and hyperlinks as edge, as Fig.2 shows. The PageRank value indicates the importance of a web page, aiming to measure the relative importance of each web page based on its parent web page's importance, namely the PageRank of all pages linking to it. And it is calculated recursively in the implementation of the algorithm. Besides, a hyperlink to this web page is called ``a vote of support for this web page". For instance, a page linked to more pages will have more votes, and also higher PageRank. 

%TODO 这不应该是垃圾邮件的那部分吗 应该在优化部分哇
In actual situations, PageRank is susceptible to be exploited, causing the fake search results. For example, some web page owners may deliberately create a large number of web pages so as to link to their own web pages, which will increase the PageRank of their web pages. Also, relevant research tends to focus on some web pages' PageRank affected by errors. In order to solve this problem, Google set up a new attribute ``nofollow'' for all web links, which allows web-masters and blogger to create some links that are not important, so that these links do not serve as the ``vote''.

\subsection{Markov Chain}

Markov chain\footnote[1]{即马尔可夫链.}, also called discrete-time\footnote[2]{即离散时间的.} Markov chain, is named after the Russian mathematician Andrei Markov. As the basis of the PageRank algorithm, it helps determine the state of each web page. Also, Markov chain has been applied to multiple statistical models.

The Markov chain is a random process that transition from a state to another in the state space\footnote[3]{即状态空间.}. The process is ``memoryless'', which means the probability distribution of the next state could merely be determined by the current state. In other words, it is irrelevant to the previous events in the time sequence. This special property ``memoryless'' is called Markov property. 

At each step of the Markov chain, the system can change from one state to another or remain unchanged, according to the probability distribution. State changes are called transitions, while the probabilities associated with different state changes are called transition probabilities.

If the Markov chain runs for enough time, the distribution of its state will reach a stationary distribution\footnote[1]{即平稳分布.}, which remains unchanged. If the Markov chain has $n$ states, the transition matrix is $P$, and the stationary distribution vector is $\boldsymbol{\pi}$, then the equation $\boldsymbol{\pi} = \boldsymbol{\pi} P$ must be satisfied.

This means that the stationary distribution vector $\boldsymbol{\pi}$ is an eigenvector\footnote[2]{即特征向量.} of the transition matrix $P$ with an eigenvalue\footnote[3]{即特征值.} of 1. A stationary distribution has the following properties:
\begin{enumerate}
	\item [1.] $\boldsymbol{\pi_i}\geq0$, and $\sum\boldsymbol{\pi_i}=1$. Each element of a stationary distribution is non-negative and sums to 1, representing probability.
	\item [2.] $\boldsymbol{\pi} = \boldsymbol{\pi} P$. The stationary distribution vector is the eigenvector of the transition matrix with an 1 eigenvalue. If it is currently stationary, then after one step of transition, the state distribution will be still stationary.
	\item [3.] \textbf{Independent to the initial state.} Whatever the initial state is, the probability distribution will utimately reach a stationary distribution.
	\item [4.] \textbf{Uniqueness}. The stationary distribution of a Markov chain is unique.
\end{enumerate}

Stationary distribution represents the stable state of the Markov chain after a series of transitions.
It measures the frequency of occurrence for each state in the entire chain, providing each proportion of time spent in the each state.

For instance, consider the process we browse web pages on the Internet. Selecting the next web page has no connection with what we browse before but depends on the current page.
It is a finite-state, and also a randomly discrete-time process.

\subsection{PageRank Algorithm and Eigenvalues of Matrix}

The PageRank algorithm is based on the matrix and the eigenvalue. Before the PageRank algorithm was proposed, eigenvalue of the matrix are used for addressing many problems, including Edmund Landau's proposal(1895) for determining the winner of a chess tournament, Gabriel Pinski and Francis Narin's work(1976) on ranking scientific journals in scientometrics, Thomas Saaty's work(1977) on the weighted alternative choice, Bradley Love and StevenSloman's work on a cognitive model, namely the centrality algorithm.

The eigenvalue of the matrix is also used by the PageRank algorithm. In the next section, we will introduce the implementation of PageRank algorithm in detail.

\section{Main Method and Theory}

\subsection{Foundational principles}

The primary implementation of the PageRank algorithm is based on the matrix operation.
Initially, we assume that there are $n$ web pages on the Internet. We can define an $n \times n$ hyperlink matrix $P$ as follows: If web page $i$ has $k (k > 0)$ hyperlinks pointing to other web pages, and if there is a hyperlink from $i$ to $j$, then $P_{ij} = 1/k$. Otherwise, $P_{ij} = 0$.

\begin{equation}
	\label{eq:1}
	Pij = 
	\begin{cases}
		\frac{1}{k} & \text{if there is a hyperlink from $i$ to $j$.} \\
		0 & \text{otherwise.}
	\end{cases}
\end{equation}

Then we can acquire the hyperlink matrix $P$ of the web pages, which is also the graph of the hyperlinked web pages.
To ensure connectivity in this graph, we assume that visitors browse randomly any web page at a certain probability. Therefore, it can be defined as stationary distribution of a Markov chain. And its state space is the set of all web pages. The transition matrix is:
\begin{equation}
	\label{eq:2}
	\widetilde{P} = cP + \frac{(1 - c)}{n}E
\end{equation}

In this equation, $E$ is a matrix that all elements equal to 1, $n$ is the number of web pages, and \(c (c \in (0,1))\) is a special parameter called \textbf{damping factor}. It represents the probability that a vistor will not finish browsing the current web page, commonly set to $0.85$\cite{ref4}.

Since the matrix $\widetilde{P}$ is a random matrix, nonperiodic\footnote[1]{即非周期性的.}, and irreducible\footnote[2]{即不可约的.}. According to the Markov chains\cite{ref5}, a unique stationary distribution vector π exists, which is an eigenvector of the transition matrix $\widetilde{P}$ with an eigenvalue of $1$\cite{ref6}. 

\begin{equation}
	\label{eq:3}
	\pi \widetilde{P} = \pi , \pi e = 1
\end{equation}

In these equations, e is a column vector with all elements equal to 1. 

A vector π satisfying equation(3) is called the \textbf{PageRank vector}. For instance, if a visitor keeps browsing a web page with probability $c$, and jumps to a random page with probability $1 - c$, then $\pi_i$ can be understood as the stationary probability of the visitor browsing the web page $i$.

Ultimately, each value of the PageRank vector represents the PageRank value of each web page. 

Besides, the PageRank values could be described using another formula.

\begin{equation}
	\label{eq:4}
	\text{PR}(A) = \frac{1 - c}{n} + c \sum_{i} \frac{\text{PR}(T_i)}{L(T_i)}
\end{equation}

In this equation, $A$ is the web page we manage to calculate the PageRank value,  $\text{PR}(T_i)$ is the PageRank value of other web page $i$ with a hyperlink pointing to $A$, and $L(T_i)$ indicates the number of hyperlinks from other web page $i$  to any other web page $j$ (See the value $k$ in Equation (1)). It is obvious that Equations (2) and (3) are the matrix form of Equation (4).

\subsection{Overview of calculating methods}

In last subsection, we describe the principles of the PageRank algorithm. In this subsection, we will introduce its calculating methods.

PageRank can be calculated by iterative method or algebraic method\footnote[1]{即代数法.}. 

The iterative method is also called power iteration. Both the two methods include the same mathematical operations. 

In the algebraic method, PageRank values are calculated by constructing and solving a group of linear equations\footnote[2]{即线性方程组.} that describes the relationships of web pages, which requires constructing a very large matrix. Solving the system of equations is much more difficult and complex, even for computers.

But for the iterative method, it could acquire final solution by several iterations. The basic idea is to use the Equation(3) to calculate the PageRank value of the web page through iterations. By multiple iterations, the PageRank values gradually verge\footnote[3]{即趋于.} to stability. This method's code is also easier to be written. Therefore, we mainly describe the iterative method in next subsection. 

\subsection{Iterative method}





We can make a citation here. \cite{ref1}

% \figurename~\ref{fig2} is a figure. You can see it at the top of the page.

% \begin{figure}[!t]
% 	\centering
% 	\includegraphics[width=2.5in]{images/fig1.png}
% 	\caption{This is a figure.}
% 	\label{fig2}
% \end{figure}

\subsection{The 3rd Section 2nd Subsection}

This is a simple subsection too.
\section{Experiment}

This is a simple section.
\subsection{The 4th Section 1st Subsection}

This is a simple subsection.

This is an equation:

You can ref it by see\eqref{eq:1}.

\subsection{The 4th Section 2nd Subsection}

This is a simple subsection too.

This is a algorithm:

\begin{algorithm}[H]
	\caption{Weighted Tanimoto ELM.}\label{alg:alg1}
	\begin{algorithmic}
		\STATE
		\STATE {\textsc{TRAIN}}$(\mathbf{X} \mathbf{T})$
		\STATE \hspace{0.5cm}$ \textbf{select randomly } W \subset \mathbf{X}  $
		\STATE \hspace{0.5cm}$ N_\mathbf{t} \gets | \{ i : \mathbf{t}_i = \mathbf{t} \} | $ \textbf{ for } $ \mathbf{t}= -1,+1 $
		\STATE \hspace{0.5cm}$ B_i \gets \sqrt{ \textsc{max}(N_{-1},N_{+1}) / N_{\mathbf{t}_i} } $ \textbf{ for } $ i = 1,...,N $
		\STATE \hspace{0.5cm}$ \hat{\mathbf{H}} \gets  B \cdot (\mathbf{X}^T\textbf{W})/( \mathbb{1}\mathbf{X} + \mathbb{1}\textbf{W} - \mathbf{X}^T\textbf{W} ) $
		\STATE \hspace{0.5cm}$ \beta \gets \left ( I/C + \hat{\mathbf{H}}^T\hat{\mathbf{H}} \right )^{-1}(\hat{\mathbf{H}}^T B\cdot \mathbf{T})  $
		\STATE \hspace{0.5cm}\textbf{return}  $\textbf{W},  \beta $
		\STATE
		\STATE {\textsc{PREDICT}}$(\mathbf{X} )$
		\STATE \hspace{0.5cm}$ \mathbf{H} \gets  (\mathbf{X}^T\textbf{W} )/( \mathbb{1}\mathbf{X}  + \mathbb{1}\textbf{W}- \mathbf{X}^T\textbf{W}  ) $
		\STATE \hspace{0.5cm}\textbf{return}  $\textsc{sign}( \mathbf{H} \beta )$
	\end{algorithmic}
	\label{alg1}
\end{algorithm}

\section{Results}

This is the results area. We should write some very nb results here.

\section{Conclusion}

This is the conclusion area. We should write a very nb conclusion here.



\begin{thebibliography}{1}

	\bibitem{ref1}
	G. Salton and M. McGill. McGraw-Hill, {\it{Introduction to modern information retrieval }}. School of Library Science University of North Carolina Chapel Hill, North Carolina NC 27514, U.S.A., 1983. 
    \bibitem{ref2}
    G. Salton, A. Wong, C. S. Yang, {\it{A vector space model for automatic indexing}}. Association for Computing Machinery, New York, NY, United States, 1975.
	\bibitem{ref3}
	JON M. KLEINBERG, {\it{Authoritative Sources in a Hyperlinked Environment}}. Cornell University, Ithaca, New York, United States, 1999.
	\bibitem{ref4}
	Sergey Brin, Lawrence Page, {\it{The Anatomy of a Large-Scale Hypertextual Web
	Search Engine}}. Computer Science Department, Stanford University, Stanford, CA 94305, USA, 1998.
	\bibitem{ref5}
	RajeevMotwani, PrabhakarRaghavan, {\it{Randomized Algorithms.}}. Cambridge University Press, United Kingdom, 1995.
	\bibitem{ref6}
	Avrachenkov K, Litvak N, {\it{Decomposition of the google pagerank and optimallinking strategy[D].}} INRIA, 2004.


\end{thebibliography}

\end{document}


