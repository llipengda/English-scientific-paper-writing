\documentclass[lettersize,journal,12pt]{IEEEtran}
\usepackage{fontspec}
\usepackage{amsmath,amsfonts}
\usepackage{algorithmic}
\usepackage{algorithm}
\usepackage{array}
\usepackage[caption=false,font=normalsize,labelfont=sf,textfont=sf]{subfig}
\usepackage{textcomp}
\usepackage{stfloats}
\usepackage{url}
\usepackage{verbatim}
\usepackage{xeCJK}
\usepackage{lettrine}
\usepackage{graphicx}
\usepackage{titling}
\usepackage{titlesec}
\usepackage{balance}
\usepackage{textcase}
\usepackage{setspace}
\setmainfont{Times New Roman}[SmallCapsFont=TeX Gyre Termes:+smcp]
% rule to break words
\hyphenation{}
\def\BibTeX{{\rm B\kern-.05em{\sc i\kern-.025em b}\kern-.08em
T\kern-.1667em\lower.7ex\hbox{E}\kern-.125emX}}
\pretitle{\begin{center}\fontsize{16}{18}\selectfont\bfseries}
\posttitle{\end{center}}
\preauthor{\begin{center}\fontsize{10}{12}\selectfont}
\postauthor{\end{center}}
\predate{\begin{center}\fontsize{10}{12}\selectfont}
\postdate{\end{center}}
\titleformat{\section}
{\filcenter\fontsize{14}{16}\bfseries\uppercase}
{\thesection}
{1em}
{}
\renewenvironment{abstract}
{\fontsize{12}{14}\textit{\textbf{\abstractname---}}\bfseries\ignorespaces}
{}
\renewenvironment{IEEEkeywords}
{\fontsize{12}{14}\textit{\textbf{Keywords---}}\bfseries\ignorespaces}{}
\begin{document}
\onehalfspacing
\title{Unveiling the PageRank Algorithm: Principles, Performance, and Enhancements}
\author{Wu Zelin, Wu Zekai, Li Pengda}

\maketitle\thispagestyle{headings}
\markboth{10225101428 吴泽霖\quad10225101429 武泽恺\quad10225101460 李鹏达}{}%

\begin{abstract}
	This is the abstract area. We should write a very nb abstract here.
\end{abstract}

\begin{IEEEkeywords}
	Keyword1, Keyword2, Keyword3
\end{IEEEkeywords}


\section{Introduction}

\subsection{Background}

\lettrine{W}{ith} 
the proliferation of the Internet technology, the explosively increasing amount of web pages on the World Wide Web has created the demand for the web searching engines with high-efficiency and high-effectiveness. 
For the biggest search engine company Google, which held a global market share of 91.54\% until November 2023\footnote[1]{Search Engine Market Share Worldwide, Statcounter GlobalStats, 2023, https:
//gs.statcounter.com/search-engine-market-share}, it is of primary significance to develop a powerful search algorithm to provide users with the most relevant and useful results in the shortest time. 

\begin{figure}[h]
    \centering
    \includegraphics[width=2.5in]{images/fig2.png}
    \caption{Google's market share in the global search engine market from December 2022 to November 2023.}
    \label{fig1}
\end{figure}

Previous studies on web search algorithms have proven that TF-IDF algorithm and BM25 algorithm were efficient and effective in the early days of the Internet.

However, these algorithms have been unable to meet the growing demand of users.
The users' search queries are becoming diversified and the web pages are becoming complex.
According to the statistics, the amount of web pages, a 130-fold increase over 20 years, has surged to 130 trillion in 2016, which means it takes much longer time to search for the most relevant web pages.
To overcome the aformentioned chanllenge, Google has developed a state-of-the-art algorithm called PageRank, which is currently a fundamental algorithm in web search, redefining how we navigate web search.

\subsection{Introduction to the PageRank Algorithm} 
PageRank is algorithm developed by Larry Page and Sergey Brin in the late 1990s to measure the importance of web pages.
It was originally created for the Google search engine and is named after one of the founder of Google Larry Page.
Google search engine uses the algorithm to analyze the relevance and importance of web pages and regard it as one of the factors to evaluate the effectiveness of web page optimization.

PageRank is an algorithm developed by Larry Page and Sergey Brin in the late 1990s to measure the importance of web pages. 
Originally created for the Google search engine, it is named after one of the founder of Google Larry Page. Except for the searching use, the Google search engine also takes advantage of this algorithm to analyze the relevance and importance of web pages, considering it as one of the factors to evaluate the effectiveness of web page optimization. 

\subsection{Outline of this Paper}

In this paper, we present the following insights and research:
\begin{itemize}
	\item The foundational principles of the PageRank algorithm.(Section 3, Subsection A)
	\item Analysis of its performance and the impact of the factors that can influence the search results and page rankings.(Section 4, Subsection A)
	\item Discussion on potential enhancements and exploration of the possibility of other factors that can improve PageRank performance.(Section 4, Subsection B)
\end{itemize}

In the next section, PageRank is briefly reviewed and analyzed.

\section{Related Work}

These section shows the brief concept of the PageRank algorithm. Theses concept motivate the design of our research.

\subsection{Synopsis of PageRank}

PageRank is a link ana1ysis algorithm and it assigns a numerical weighting to each element of hyperlinked set of documents. It is aimed to achieve the purpose of measuring the relative importance of a element of a collection. The algorithm can be applied to any collection containing mutual references between elements.We call the weight value of any element E as ``PageRank of E", which is represented symbolically as $\boldsymbol{PR(E)}$. Other factors such as ``Author Rank" can also affect the weight value of an element.

A PageRank results from a mathematical algorithm based on the webgraph, created by all World Wide Web pages as nodes and hyperlinks as edge, taking into consideration anthority hubs such as CNN. The rank value indicates an importance of a particular page. A hyperlink to this web page is called ``a vote of support for this web page". The weight of each web is defined recursively, based on the weight of all pages linking to it. For example, a page linked to many pages will have a high PageRank.



\subsection{Markov Chain}

This is a simple subsection too.

\section{Main Method and Theory}



\subsection{Subsection 1}

This is a simple subsection.
We can make a citation here. \cite{ref1}

\figurename~\ref{fig2} is a figure. You can see it at the top of the page.

\begin{figure}[!t]
	\centering
	\includegraphics[width=2.5in]{images/fig1.png}
	\caption{This is a figure.}
	\label{fig2}
\end{figure}

\subsection{The 3rd Section 2nd Subsection}

This is a simple subsection too.
\section{Experiment}

This is a simple section.
\subsection{The 4th Section 1st Subsection}

This is a simple subsection.

This is an equation:

\begin{equation}
	\label{eq:1}
	e^{\pi i} + 1 = 0
\end{equation}
You can ref it by see\eqref{eq:1}.

\subsection{The 4th Section 2nd Subsection}

This is a simple subsection too.

This is a algorithm:

\begin{algorithm}[H]
	\caption{Weighted Tanimoto ELM.}\label{alg:alg1}
	\begin{algorithmic}
		\STATE
		\STATE {\textsc{TRAIN}}$(\mathbf{X} \mathbf{T})$
		\STATE \hspace{0.5cm}$ \textbf{select randomly } W \subset \mathbf{X}  $
		\STATE \hspace{0.5cm}$ N_\mathbf{t} \gets | \{ i : \mathbf{t}_i = \mathbf{t} \} | $ \textbf{ for } $ \mathbf{t}= -1,+1 $
		\STATE \hspace{0.5cm}$ B_i \gets \sqrt{ \textsc{max}(N_{-1},N_{+1}) / N_{\mathbf{t}_i} } $ \textbf{ for } $ i = 1,...,N $
		\STATE \hspace{0.5cm}$ \hat{\mathbf{H}} \gets  B \cdot (\mathbf{X}^T\textbf{W})/( \mathbb{1}\mathbf{X} + \mathbb{1}\textbf{W} - \mathbf{X}^T\textbf{W} ) $
		\STATE \hspace{0.5cm}$ \beta \gets \left ( I/C + \hat{\mathbf{H}}^T\hat{\mathbf{H}} \right )^{-1}(\hat{\mathbf{H}}^T B\cdot \mathbf{T})  $
		\STATE \hspace{0.5cm}\textbf{return}  $\textbf{W},  \beta $
		\STATE
		\STATE {\textsc{PREDICT}}$(\mathbf{X} )$
		\STATE \hspace{0.5cm}$ \mathbf{H} \gets  (\mathbf{X}^T\textbf{W} )/( \mathbb{1}\mathbf{X}  + \mathbb{1}\textbf{W}- \mathbf{X}^T\textbf{W}  ) $
		\STATE \hspace{0.5cm}\textbf{return}  $\textsc{sign}( \mathbf{H} \beta )$
	\end{algorithmic}
	\label{alg1}
\end{algorithm}

\section{Results}

This is the results area. We should write some very nb results here.

\section{Conclusion}

This is the conclusion area. We should write a very nb conclusion here.



\begin{thebibliography}{1}

	\bibitem{ref1}
	S. Zhan, S. Li and W. Wang, {\it{A Very Nb Book}}. Shanghai, P.R.C., East China Normal  Univ. Press, 2022.

\end{thebibliography}

\end{document}


